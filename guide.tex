\documentclass{article}
\title{Inference Study Guide [Still Under Construction]}
\author{Sean Carver}
\begin{document}
\maketitle
\section{The book and the test}
Our textbook, De Veaux, has a fantastic presentation of the theory and
concepts behind inference.  Many of this theory and concepts are
important for applying inference to problems.  However, I have found
over the past couple of semesters that I have used the book, that
students don't always master the concepts and theory, and many of
them, at the same time, do not learn to apply the methods to problems.

We will continue to go through the book as far as we can---and do the
MyStatLab homework---but my intention is to provide problems ahead of
time that you will need to solve to do well on the final exam.  To
this extent, I have posted the midterm I used for the last semester.
This is a study guide for learning to master those problems.

\section{Lists of Concepts}
The ``stats'' on the menu in StatCrunch allow you to perform inference
in two different ways:
  \begin{enumerate}
  \item Confidence intervals.
  \item Hypothesis tests.
  \end{enumerate}
There are 5 general types of ``stats'' (what StatCrunch labels the
menus) that we will cover (the top two submenus)
\begin{enumerate}
\item One-sample proportion stats.
\item One-sample t-stats.
\item Two-sample proportion stats.
\item Two-sample t-stats.
\item Paired t-stats.
\end{enumerate}
For confidence interval, you need to know:
\begin{enumerate}
\item Confidence level.
\item Point estimate.
\item Standard error.
\item Critical value of test statistic.
\item Margin of error.
\item Maybe more stuff, stay tuned...
\end{enumerate}
For hypothesis testing, you need to know:
\begin{enumerate}
\item P-value.
\item Test statistic.
\item Critical value of test statistic.
\item Level of significance of test.
\item One-sided alternative test.
\item Two-sided alternative test.
\item Type I error.
\item Type II error.
\item Power of test.
\item Maybe more stuff, stay tuned...
  \end{enumerate}
The types of problems you will encounter on the exam include:
\begin{enumerate}
\item Given a scenario, choose the type of ``stat'' that you will need
  to employ from the StatCrunch menu.
\item Given a problem, check the conditions for the type of ``stat''
  you need to use.
\item Given a problem, plug the numbers into StatCrunch (choose ``with
  data'' or ``with summary'' appropriately, and fill out dialog box,
  correctly).
\item Interpret the given output of StatCrunch for an inference problem.
\item Do all the steps above in one problem.
\item Solve problems that involve just concepts and theory in addition
  to the problems above that may also require knowing concepts and
  theory.
\end{enumerate}
\textbf{Please see the practice exam for examples.}
\end{document}
